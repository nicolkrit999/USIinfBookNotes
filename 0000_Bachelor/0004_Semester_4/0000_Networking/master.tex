\documentclass[a4paper]{report}
\input{header.tex}
\author{Krit Pio Nicol}
\title{Note Template}

\thispagestyle{empty}
\addbibresource{ref.bib}

\usepackage{adjustbox}
\usepackage{centernot}



\begin{document}



%%%%%%%%%%% TITLE PAGE

\begin{titlepage}
\begin{figure}[t]
    \centering\includegraphics[width=0.5\textwidth]{Figures/usilogo.png}
\end{figure}

\begin{center}
    \textsc{\LARGE{USI Lugano\\}}
	\textsc{ \LARGE{Computer Science Department\\ }}
	%\textnormal{ \LARGE{Corso di Laurea Triennale/Magistrale in ???\\}}
	\vspace{40mm}
	\fontsize{10mm}{7mm}\selectfont 
    \textup{Computer Networking book}\\
\end{center}

\vspace{25mm}

\begin{minipage}[t]{0.47\textwidth}
	\textnormal{\large{\bf Professor\\}}
	{\large Prof. Santini\\ \\}
 \textnormal{\large{\bf TA's\\}}
	{\large M. Laporte, L. Alecci}
\end{minipage}\hfill\begin{minipage}[t]{0.47\textwidth}\raggedleft
	\textnormal{\large{\bf Student\\}}
	{\large Krit Pio Nicol}
\end{minipage}

\vspace{20mm}

\centering{\large{Year 2024 }}

\end{titlepage}

\begin{abstract}

    \begin{itemize}
        \item The Internet is the largest engineered system in history, comprising millions of computers and communication devices, billions of users, and a vast array of Internet-connected "things".
        \item Despite its size and complexity, it is possible to understand how the Internet works, thanks to guiding principles and structures.
        \item The aim is to provide a modern introduction to computer networking, emphasizing principles and practical insights for understanding current and future networks.
        \item This overview includes:
        \begin{itemize}
            \item Basic terminology and concepts in computer networking.
            \item Examination of network hardware and software components, including end systems, network applications, links, switches, access networks, and physical media.
            \item Understanding the Internet as a network of networks and how these networks interconnect.
            \item Discussion on the core aspects of computer networks, such as delay, loss, and throughput, including simple quantitative models for end-to-end throughput and delay.
            \item Key architectural principles in computer networking, including protocol layering and service models.
            \item Overview of security challenges and types of attacks in computer networks, with strategies for enhancing security.
            \item A brief history of computer networking to close the chapter.
        \end{itemize}
        \item The chapter sets the context for the rest of the book, aiming to see "the forest through the trees" by providing a broad picture without losing sight of the detailed aspects of computer networking.
    \end{itemize}
    
\end{abstract}

\newpage

\tableofcontents


%─────Formatting_References───────────────────────────────────────────────────────────────────────────────────────────────────────────────────────────
\input{Lectures/0001_Examples_1}
\input{Lectures/0002_Examples_2}

%─────Lectures{start{end}─────────────────────────────────────────────────────────────────────────────────────────────────────────────────────────────
\lec{1}{9999}

\newpage
%─────Appendix────────────────────────────────────────────────────────────────────────────────────────────────────────────────────────────────────────
%\appendix
%\appendixpage

%\input{appendix.tex}

\newpage
%─────Reference───────────────────────────────────────────────────────────────────────────────────────────────────────────────────────────────────────
%\printbibliography

\end{document}