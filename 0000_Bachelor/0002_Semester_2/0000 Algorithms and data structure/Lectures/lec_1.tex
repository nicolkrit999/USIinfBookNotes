\chapter{The role of algorithms in computing (1-38)}


\lecture{1}{19 Feb}{1-27}

\section{Preface and introduction}
Not relevant

\lecture{2}{21 Feb}{27-54}

\section{Algoriths}
\subsection{Data structures}

\begin{definition}[Data structure]\label{def:data_structure_1}
    A way to store and organize data in order to facilitate access and modifications.
\end{definition}

\begin{definition}[Efficiency]\label{def:efficiency_1}
    Different algorithms solve the same problem but have different level of efficiency.
    
\end{definition}

\begin{eg}[Insertion sort vs merge sort]\label{eg:insertion_sort_vs_merge_sort_1}
  We note that the two sorting algorithms have different efficiency
  \begin{note}[Insertion sort]\label{note:insertion_sort_1}
    $C_1*n^2$ where $C_1$ is constant independent of n
  \end{note}

  \begin{note}[Merge sort]\label{note:merge_sort_1}
    $C_2 * n * \log_{2}n$ where $C_2$ is constant independent of n
  \end{note}

  \begin{remark}[Constant factor]\label{def:constant_factor_1}
    insertion sort has a smaller constant factor than merge sort $C_1 < C_2$. The majority of the times constant factor has less influence than input size $n$
  \end{remark}

  \begin{definition}[Constant factor]\label{def:constant_factor_2}
      Anything that doesn't depend on the input parameter(s)
  \end{definition}

  \begin{definition}[Input size]\label{def:constant_factor_2}
      The input size can be the following:

      \begin{itemize}
          \item Number of items in the input, eg the number of items to sort
          \item Total number of bits, eg bitwise multiplication to multiply 2 integers
          \item Input size in term of 2 numbers, eg for finding the shortest path in a graph from a given source
      \end{itemize}
  \end{definition}

  \end{eg}

  \chapter{Getting started (39-49}

  \section{Insertion sort}
  \begin{definition}[keys]\label{def:keys_1}
      The numbers to be sorted. The input comes in the form of an array with $"n"$ elements. The keys are often associated with other data, called "satellite data". Together they form a "record" 
  \end{definition}

  \begin{definition}[Arrays]\label{def:arrays_1}
      A data structure. A collection of items stored at contiguous memory locations. The idea is to store multiple items of the same type together. This makes it easier to calculate the position of each element by simply adding an offset to a base value, i.e., the memory location of the first element of the array (generally denoted by the name of the array).
  \end{definition}

\begin{eg}[How it works]
To sort an array of size N in ascending order iterate over the array and compare the current element (key) \hyperref[def:keys_1]{keys} to its predecessor, if the key element is smaller than its predecessor, compare it to the elements before. Move the greater elements one position up to make space for the swapped element.
\end{eg}

